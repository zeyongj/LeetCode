\documentclass{article}

\title{LeetCode 485: Max Consecutive Ones}
\author{Zeyong Jin, Simon Fraser University}
\date{\today}

\usepackage{algorithm2e}
\begin{document}

\maketitle

Here I presented a pseudo-code to solve the 485th question of LeetCode: Max Consecutive Ones.

The description of this question is that given a binary array \texttt{nums}, return the maximum number of consecutive 1's in the array.

The constranits are:
\begin{enumerate}
   \item 1 $\leq$ \texttt{nums}.length $\leq$ $10^{5}$.
   \item \texttt{nums}[i] is either 0 or 1.
\end{enumerate}


The pseudo-code is as follows:

\RestyleAlgo{ruled}
%% This is needed if you want to add comments in
%% your algorithm with \Comment
\SetKwComment{Comment}{/* }{ */}

\begin{algorithm}[hbt!]
\caption{Max Consecutive Ones}\label{alg:two}
\LinesNumbered
\KwIn{int nums[]}
\KwOut{int ans}
$count \gets 0$\;
$result \gets 0$\;
$ans \gets 0$\;
$size \gets length(nums)$\;
\eIf{$size = 0$}{
    return ans\Comment*[r]{ans = 0}
  }
  {
  $i \gets 0$\;
  \While{$i < size$}{
  \eIf{nums[i] == 1}{
    $count \gets count + 1$\;
  }{
      $result \gets max(result, count)$\;
      $count \gets 0$\;
  }
}
$ans \gets max(result, count)$\;
return ans\;
  }

\end{algorithm}


\end{document}
