\documentclass{article}
\usepackage[T1]{fontenc}
\usepackage[utf8]{inputenc}
\usepackage{hyperref}
\usepackage{authblk}
\setlength{\parindent}{2em}
\setlength{\parskip}{1em}
\renewcommand{\baselinestretch}{1}

\title{LeetCode 283: Move Zeroes}
\author{Zeyong Jin\thanks{zeyongj@gmail.com}}
\affil{School of Computing Sciences, Simon Fraser University}
\date{\today}

\usepackage{algorithm2e}
\hypersetup{
    colorlinks=true,
    linkcolor=blue,
    filecolor=magenta,      
    urlcolor=cyan,
    pdftitle={Overleaf Example},
    pdfpagemode=FullScreen,
    }
\urlstyle{same}
\usepackage{enumitem}
\begin{document}

\maketitle

Here I presented a pseudo-code to solve the 283rd question of LeetCode: Move Zeroes.

The description of this question is that given an integer array  \texttt{nums}, move all 0's to the end of it while maintaining the relative order of the non-zero elements.

One \textbf{note} of this question is that you must do this in-place without making a copy of the array.


The constranits of this question are as follows:
\begin{enumerate}[label=(\alph*),leftmargin=2\parindent]
   \item 1 $\leq$ \texttt{nums}.length $\leq$ $10^{4}$.
   \item $-2^{31}$ $\leq$ \texttt{nums}[i] $\leq$ $2^{31} - 1$.
\end{enumerate}

One \textbf{follow up} question is that whether you could minimize the total number of operations done.

For more information about this question, click on the following link: 
    \href{https://leetcode.com/problems/move-zeroes/}{Move Zeroes} or go to the next url: \url{https://leetcode.com/problems/move-zeroes/}.

The pseudo-code is in the following page.

The time complexity of this algorithm is $O(n)$.

The space complexity of this algorithm is $O(1)$.


\RestyleAlgo{ruled}
%% This is needed if you want to add comments in
%% your algorithm with \Comment
\SetKwComment{Comment}{/* }{ */}

\begin{algorithm}[hbt!]
\caption{Move Zeroes}\label{alg:two}
\LinesNumbered
\KwIn{int\& nums}
\KwOut{none}
$cursor \gets 0$\;
$size \gets length(nums)$\;
$i \gets 0$\;
\While{$i < size$}{
  \eIf{$nums[i]$ != 0}{
    $nums[cursor]$ = $nums[i]$\;
    $cursor$++ \Comment*[r]{Cursor move to the next index waiting to be exchanged.}
  }{$continue$ ;
  }
}
$i \gets cursor$\ \Comment*[r]{Cursor now at the index next to the final position of non-zero value, i.e. cursor now points at the index of the first zero value.}
\While{$i < size$}{
  $nums[i]$ = 0 \Comment*[r]{Starting from the cursor, all values would be 0s.}
}

\end{algorithm}

\end{document}
